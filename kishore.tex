\documentclass[12pt,conference,onecolumn]{IEEEtran}

\usepackage[hidelinks]{hyperref}

\title{Lasers are fun}
\author{%
\IEEEauthorblockN{Nitika Kishore}\IEEEauthorblockA{Science \& Engineering\\Manalapan High School\\Englishtown, NJ\\\href{mailto:426nkishore@frhsd.com}{426nkishore@frhsd.com}}\and
\IEEEauthorblockN{Dev Parekh}\IEEEauthorblockA{Science \& Engineering\\Manalapan High School\\Englishtown, NJ\\\href{mailto:426dparekh@frhsd.com}{426dparekh@frhsd.com}}}
\date{June 16, 2026}

\newcommand{\keywords}{lasers}

\usepackage{hyperref}
\makeatletter
\AtBeginDocument{
\hypersetup{%
pdftitle={\@title},
pdfauthor={Nitika Kishore and Dev Parekh},
pdfkeywords={\keywords}}}
\makeatother

\begin{document}
\maketitle 

\begin{abstract}
This project will research on how lasers work, delving deeper into the physics of the light movement and coherence along with how lasers are implemented in modern technology. Using the research gathered, this project will explore several prototypes for laser-based applications. One such prototype will be a mirror-controlled laser pathway that projects geometrics shapes on a wall by moving the mirrors in certain angles and will eventually project images that react to sound and beats in music. Another prototype is a laser distance measurement device that incorporates laser angles and reflection to record the space between two objects. Primary materials that will be used for this project would be a low powered, class 1 or class 2, laser along with mirrors, servo motors, arduinos, and basic electronic material. This project will aim to combine physics and mathematical knowledge with the application of lasers and demonstrate how they can be used in professional and real-world engineering settings. 
\end{abstract}

\begin{IEEEkeywords}
\keywords
\end{IEEEkeywords}

\end{document}
